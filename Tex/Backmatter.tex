%%
%%% >>> Resume and Published papers
%%
\chapter{作者简历}

\section*{作者基本情况}

刘志聪,男,1992年出生于河北省邢台市平乡县,未婚,中国科学院高能物理研究所博士研究生。

\section*{教育情况}
\noindent
2009年9月到2013年7月,天津大学,本科。

专业:应用物理学。

\noindent
2013年9月到2018年7月,中国科学院高能物理研究所,博士研究生。

专业:粒子物理与原子核物理,专业方向:加速器物理。

导师:秦庆,研究员。

\noindent
2016年8月到2018年3月,美国劳伦斯伯克利国家实验室,联合培养博士研究生。

方向:加速器并行模拟软件开发。

合作导师:Ji Qiang, Senior Scientist.

\section*{联系方式}

通讯地址:北京市石景山区玉泉路19号(乙)918信箱,中国科学院高能物理研究所

邮编:100049

E-mail: liuzhicong@ihep.ac.cn

\chapter{攻读学位期间发表的学术论文及科研成果}
\begin{enumerate}
  \item \textbf{\textit{Liu, Z.}}, Qiang J., Symplectic Multi-Particle Tracking Using GPU, Computer Physics Communications. \color{red}Accepted\color{black}.
  \item \textbf{\textit{Liu, Z.}}, Li C., Qin Q., Structure resonance crossing in space charge dominated beams, Physical Review Accelerators and Beams. \color{red}Under review\color{black}.
  \item \textbf{\textit{Liu, Z.}}, Qiang J., Implementation of a Beam Dynamics PIC Code on Hybrid Computer Architectures, Nuclear Instruments and Methods in Physics Research Section A: Accelerators, Spectrometers, Detectors and Associated Equipment. \color{red}Plan to submit\color{black}.
  \item Li C., \textbf{\textit{Liu Z.}}, Zhao Y., Qin Q., Nonlinear resonance and envelope instability of intense beam in axial symmetric periodic channel. Nuclear Instruments and Methods in Physics Research Section A: Accelerators, Spectrometers, Detectors and Associated Equipment, 813, 13-18.
  \item \textbf{\textit{Liu, Z.}}, Li, C., Qin, Q., Zhao, Y., Yan, F., Beam Dynamics Study of C-Ads Injector-I With Developing P-Topo Code. In 57th ICFA Advanced Beam Dynamics Workshop on High-Intensity and High-Brightness Hadron Beams (HB'16), Malmö, Sweden, July 3-8, 2016 (pp. 195-198).
  \item \textbf{\textit{Liu, Z.}}, Qiang, J. Symplectic Multi-Particle Tracking Using Cuda. In 8th Int. Particle Accelerator Conf.(IPAC'17), Copenhagen, Denmark, 14-19 May, 2017 (pp. 3756-3759).
  \item Yu, C., Duan, Z., Gu, S., Guo, Y., Huang, X., Ji, D., Ji H., Jiao Y.,\textbf{\textit{Liu, Z.}} ... \& Qin, Q. (2016, June). BEPCII performance and beam dynamics studies on luminosity. IPAC16, Busan, Korea, May 8-13, 2016 (pp. 1014-1018).
\end{enumerate}



\section*{项目资助情况}

%%可以随意添加新的条目或是结构
Supported by the Ministry of Science and Technology of China under Grant no.2014CB845501.
%%
%%% >>> Acknowledgements
%%
\chapter{致\quad 谢}

值此论文完成之际,谨在此向多年来给予我关心和帮助的老师、学长、同学、
朋友和家人表示衷心的感谢!


