%%
%%% >>> Resume and Published papers
%%
\chapter{作者简历}

\section*{作者基本情况}

刘志聪,男,1992年出生于河北省邢台市平乡县,未婚,中国科学院高能物理研究所博士研究生。

\section*{教育情况}
\noindent
2009年9月到2013年7月,天津大学,本科。

专业:应用物理学。

\noindent
2013年9月到2018年7月,中国科学院高能物理研究所,博士研究生。

专业:粒子物理与原子核物理,专业方向:加速器物理。

导师:秦庆,研究员。

\noindent
2016年8月到2018年3月,美国劳伦斯伯克利国家实验室,联合培养博士研究生。

方向:加速器并行模拟软件开发。

合作导师:Ji Qiang, Senior Scientist.

\section*{联系方式}

通讯地址:北京市石景山区玉泉路19号(乙)918信箱,中国科学院高能物理研究所

邮编:100049

E-mail: liuzhicong@ihep.ac.cn

\chapter{攻读学位期间发表的学术论文及科研成果}
\begin{enumerate}
  \item \textbf{\textit{Liu, Z.}}, Qiang J., (2018). Symplectic multi-particle tracking on GPUs. Computer Physics Communications. \href{https://doi.org/10.1016/j.cpc.2018.02.001}{https://doi.org/10.1016/j.cpc.2018.02.001}
  \item \textbf{\textit{Liu, Z.}}, Li C., Qin Q., Structure resonance crossing in space charge dominated beams, Physical Review Accelerators and Beams. \color{red}Under review\color{black}.
  \item \textbf{\textit{Liu, Z.}}, Qiang J., Implementation of a Beam Dynamics PIC Code on Hybrid Computer Architectures, Nuclear Instruments and Methods in Physics Research Section A: Accelerators, Spectrometers, Detectors and Associated Equipment. \color{red}Under review\color{black}.
  \item Li C., \textbf{\textit{Liu Z.}}, Zhao Y., Qin Q., Nonlinear resonance and envelope instability of intense beam in axial symmetric periodic channel. Nuclear Instruments and Methods in Physics Research Section A: Accelerators, Spectrometers, Detectors and Associated Equipment, 813, 13-18.
  \item \textbf{\textit{Liu, Z.}}, Li, C., Qin, Q., Zhao, Y., Yan, F., Beam Dynamics Study of C-Ads Injector-I With Developing P-Topo Code. In 57th ICFA Advanced Beam Dynamics Workshop on High-Intensity and High-Brightness Hadron Beams (HB'16), Malmö, Sweden, July 3-8, 2016 (pp. 195-198).
  \item \textbf{\textit{Liu, Z.}}, Qiang, J. Symplectic Multi-Particle Tracking Using Cuda. In 8th Int. Particle Accelerator Conf.(IPAC'17), Copenhagen, Denmark, 14-19 May, 2017 (pp. 3756-3759).
  \item Li, C., Zhao, Y.,\textbf{\textit{Liu, Z.}},  Qin, Q. Space Charge Induced Collective Modes and Beam Halo in Periodic Channels. In 8th Int. Particle Accelerator Conf.(IPAC'16), Busan, Korea, May 8-13, 2016 (pp. 3165-3166).
  \item Yu, C., Duan, Z., Gu, S., Guo, Y., Huang, X., Ji, D., Ji H., Jiao Y.,\textbf{\textit{Liu, Z.}} ... \& Qin, Q. BEPCII performance and beam dynamics studies on luminosity. In 8th Int. Particle Accelerator Conf.(IPAC'16), Busan, Korea, May 8-13, 2016 (pp. 1014-1018).
\end{enumerate}



\section*{项目资助情况}

%%可以随意添加新的条目或是结构
Supported by the Ministry of Science and Technology of China under Grant no.2014CB845501.
%%
%%% >>> Acknowledgements
%%
\chapter{致\quad 谢}

值此论文完成之际,谨在此向多年来给予我关心和帮助的老师、同学、朋友和家人表示衷心的感谢!

%秦老师
首先,我深深地感谢导师秦庆研究员对我的悉心指导。
秦老师开拓性的科研抱负,严谨的学术态度,以及掌控大局的学术思维,都使我受益匪浅。
老师的启发和鼓励帮助我克服了很多困难,回顾自己在科研领域的进展,每一步都凝聚了老师无数的汗水。
在学生们的印象中,老师是一位和蔼、儒雅的知识分子,更是一个很少休息和一心扑在科研事业上的“工作狂”。
值此毕业之际,谨向老师表示崇高的敬意。

%伯克利,墙老师
感谢国家留学基金委资助我前往美国劳伦斯伯克利国家实验室交流学习,
感谢在伯克利期间的导师墙棘(Ji Qiang)研究员对我的教导和帮助。
墙老师不但在学术上给予了我无微不至的教导,还在我初到美国时在生活上给予了我很大的帮助和关怀。
墙老师谦和的人品、渊博的知识、前沿而精髓的学术造诣、严谨的治学态度以及理论与实际相结合的学术风格,
都让我永志难忘,深刻影响着我日后的工作和生活。

%同学,师兄弟,李超
感谢高能所物理组的同学和师兄弟们,
感谢高能所我的舍友和朋友们,
感谢伯克利实验室和辅友的伙伴们,
和他们在一起的日子让我收获颇多,与他们的讨论不时激发我的灵感。
他们的存在使原本枯燥的科研生活变得丰富多彩,感谢你们在我身旁鼓励、帮助我。
其中,我需要特别感谢李超师兄,他在我的博士期间在工作和生活上都充当着兄长的角色,
对我的学习和生活给予了相当大的关心与支持。

%家人
感谢我的父母和家人一直依赖对我的支持和鼓励,让我可以在没有过多经济压力的条件下专心向学。
父母经常叮嘱我要认真做事、踏实做人,努力去做更好的自己,
给了我一个刻苦拼搏、积极向上的家庭环境,对我的学习工作以及整个人生成长起到了巨大的影响。
感谢我的女友王卫娟女士多年来对我关怀体贴,她一如既往的支持我的学业。
在我无心向学的时候,是她不断的鼓励我,帮助我调整心态。
没有女友在后台的默默支持,要顺利的完成论文的研究工作将会更加艰难。

%总结
感谢曾给过我教诲的所有师长。感谢所有曾经帮助过我的人。也要感谢自己的坚持与努力。
在今后的工作中我会加倍的努力,希望我的学术道路能够永远更进一步。