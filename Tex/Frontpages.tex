%%
%%% >>> Title Page
%%
%%
%%% Chinese Title Page
%%
  \confidential{}% show confidential tag
  \schoollogo{scale=0.112}{UCAS}% university logo
  \title[强流束流动力学研究与软件开发]{强流束流动力学研究与软件开发}% \title[short title for headers]{Long title of thesis}
  \author{刘志聪}% name of author
  \advisor{秦庆~研究员}% names and titles of supervisors
  \advisorinstitute{中国科学院高能物理研究所}% institute names of supervisors
  \degree{博士}% degree
  \degreetype{理学}% degree type
  \major{粒子物理与原子核物理}% major
  \institute{中国科学院高能物理研究所}% institute of author
  \chinesedate{2018~年~06~月}% only need for user customized date
%%
%%% English Title Page
%%
  \englishtitle{Study of High Current Beam Dynamics \\and\\ the Development of Simulation Code}
  \englishauthor{Zhicong Liu}
  \englishadvisor{Professor Qing Qin}
  \englishdegree{Doctor}
  \englishthesistype{thesis}
  \englishmajor{Nuclear and Particle Physics}
  \englishinstitute{Institute of High Energy Physics, Chinese Academy of Sciences}
  \englishdate{June, 2018}% only need for user customized date
%%
%%% Generate Chinese Title
%%
\maketitle
%%
%%% Generate English Title
%%
\makeenglishtitle
%%
%%% >>> Author's declaration
%%
\makedeclaration
%%
%%% >>> Abstract
%%
\chapter{摘\quad 要}% does not show the title on the top
%\begin{abstract}% will show the title on the top

%theory
强流束流动力学主要问题是空间电荷效应问题。空间电荷效应会导致束流和粒子的共振和不稳定性,束流的共振和不稳定的一个显著结果就是束晕,从而产生束损。为了探寻束晕产生背后的物理机制,我们使用了非自洽的模型对束晕产生的机制进行了分析,得到了一个粗略的物理图像。由于空间电荷效应问题是一个多体耦合问题,我们无法得到解析解,为了更加深入的了解问题背后的机制,我们需要一个软件进行精确的数值模拟。

%PIC code
束流模拟软件在加速器建设中起到了必不可少的作用。随着束流流强的增大,需要模拟的粒子数目增加了几个量级,需要的功能也越来越多。面对新的物理问题,我们需要探索新的算法,优化新的程序结构,编写出一个新的粒子模拟软件,了解并掌握软件背后的计算过程。
粒子云网格(Particle-In-Cell)算法被广泛应用于加速器的研究和设计中。基于此,作者在前人的基础上使用C++开发了束流模拟软件P-TOPO。并且对ADS进行了end-to-end模拟,模拟结果和TraceWin进行对照相符很好。作者还编写CPU和GPU两个版本,实现并行计算,来提高模拟软件的效率。

%Symplectic code
除此之外,作者还编写了基于GPU的求解空间电荷效应的无网格保辛算法。这种算法模型并不利用网格,而是利用高阶分解来求解空间电荷效应。相比PIC算法,这种方法能够显著的降低由于网格数值噪声带来的发射度增长,但是计算量要大得多。然而由于无网格算法很适合并行,有很好的可扩展性,这种算法非常适合使用GPU进行加速计算。通过使用CUDA和GPU,可以显著加快无网格粒子跟踪代码的运行速度。

\keywords{空间电荷效应,PIC,Symplectic,GPU,模拟软件}
%\end{abstract}


\chapter{Abstract}% does not show the title on the top
%\begin{englishabstract}% will show the title on the top

%theory
A main problem of high current beam dynamics is space charge effect, which would lead to the resonance and instability of beam and particles. A direct result from resonance is beam halo and lose. 
In order to explore the physical mechanism behind the halo formation, we used a non-self-consistent model to analyze the propagation of beam and obtained a rough physical image.
Since the problem of space charge effect is a multi-body coupling problem, we can not get the analytic solution. In order to understand the mechanism behind the problem, we need a software to carry on the accurate numerical simulation.

%PIC code
Beam simulation code plays an indispensable role in accelerator construction. 
As the beam current increases, the number of particles to be simulated increases by several orders of magnitude 
and more features are required. 
Faced with new physical problems, we need to explore new algorithms, optimize new program structure, and make a new particle simulation software.
It also help us understanding the algorithms behind the calculation process.
Particle-In-Cell (PIC) algorithm is widely used in accelerator research and design. 
Based on PIC algorithm, the author developed a beam simulation software P-TOPO using C++ on the basis of predecessors. 
An end-to-end simulations of ADS was done, and the simulation results were in good agreement with TraceWin.
The author also prepared code in CPU and GPU versions to achieve parallel computing and improve the efficiency of simulation software.

%Symplectic code
In addition, the author also made a GPU-based the space charge effect solver based on gridless symplectic algorithm. 
This model uses a gridless spectral method instead of mesh point to calculate the space charge effect. 
It can effectively reduce the emittance growth associated with numerical grid heating compared with the PIC algorithm.
However, this model is much slower compared with PIC method.
Fortunately, it is very suitable for parallelism and can achieve very good speeded and scalability, especially by using GPU and CUDA library

\englishkeywords{Space Charge Effect, PIC, Symplectic, GPU, Simulation Code}
%\end{englishabstract}
