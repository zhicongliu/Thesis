
\chapter{强流束流动力学}
\label{chap:AccBasic}

\section{基本理论}
束流动力学的基础是电动力学\cite{jackson1975electrodynamics,guo1979electrodynamics}。
一个电荷为$e$的带电粒子,在电场强度为$\vec{E}$、磁感应强度为$\vec{B}$的电磁场中运动,其运动方程为:
\begin{equation}
    \label{eq:Newton}
    \frac{d \vec{p}}{dt}=e(\vec{E}+\vec{v} \times \vec{B}),
\end{equation}
其中,$\vec{p}=\gamma m \vec{v}$为粒子的动量,
$\vec{v}$是带电粒子的速度,
$m$为粒子的固有质量,
$\gamma$为洛伦兹因子:
\begin{equation}
    \label{eq:Lorentz}
    \gamma = \frac{1}{\sqrt{1-\frac{v^2}{c^2}}}, \qquad v = \sqrt{\vec{v} \cdot \vec{v}},
\end{equation}
这里$c$为光速。

电磁场满足麦克斯韦方程组。在真空中,麦克斯韦方程可以表达为:
\begin{equation}
    \label{eq:Maxwell}
    \begin{aligned}
    \nabla \cdot \vec{E}  &= \frac{\rho}{{\varepsilon}_{0}}, \\
    \nabla \cdot \vec{B}  &= 0, \\
    \nabla \times \vec{E} &= -\frac{\partial \vec{B}}{\partial {t}}, \\
    \nabla \times \vec{B} &= {\mu}_{0}\left(\vec{J}+\varepsilon \frac{\partial \vec{E}}{\partial {t}}\right) \text{。}
    \end{aligned}
\end{equation}
其中,$\rho$为电荷密度,$\varepsilon _0$为真空介电常数,${\mu}_{0}$为真空磁导率,$\vec{J}$为电流密度。

\section{束流横向运动}
束流的横向运动已经得到了广泛研究\cite{qinq2011ring,accelerator2004lee,accelerator2013chao}。
我们定义束流横向运动与纵向运动方向的夹角为$x'$:
\begin{equation}
    \label{eq:transverse1}
    x'=\frac{dx}{dz}=\frac{p_x}{p_z},
\end{equation}
则
\begin{equation}
    \label{eq:transverse2}
    F_x =\frac{dp_x}{dt}=\frac{dp_x}{dz}\frac{dz}{dt}=v_z \frac{d(p_z x')}{dz}=v_z(p_z x''+p_z'x'),
\end{equation}
假设束流没有被加速,即$p_z'=0$,则粒子的横向运动方程为:
\begin{equation}
    \label{eq:transverse3}
    x'' = \frac{F_x}{p_z v_z} \text{。}
\end{equation}

\begin{figure}[!tbh]
    \centering
    \begin{subfigure}[b]{0.45\textwidth}
        \includegraphics[width=\textwidth]{Img/Quad_B.pdf}
    \end{subfigure}
    \qquad
    \begin{subfigure}[b]{0.4\textwidth}
        \includegraphics[width=\textwidth]{Img/Quad_F.pdf}
    \end{subfigure}
    %\includegraphics[width=0.99\textwidth]{Img/quadrupole.pdf}
    \caption{四极磁铁中磁场及粒子受力示意图}
    \label{fig:quadrupole}
\end{figure}
现代加速器中一般使用二极磁铁来对束流进行偏转,使用四极磁铁来对束流进行横向聚焦。
四极磁铁的磁场分布和粒子受力如图\eqref{fig:quadrupole}所示\cite{qinq2011ring},其中带负电荷的粒子垂直于纸面向里运动。
可以看出,四极铁产生的力在一个方向上聚焦,而在另一个方向上散焦。
粒子所受到的力与粒子的位置成正比。以X方向为例,粒子所受到的力为:
\begin{equation}
    \label{eq:quadrupole_force}
    F_x = -q {v}_{z} {B}_{y} = -q {v}_{z} g x, \qquad g=\frac{\partial B_y}{\partial x}\text{。}
\end{equation}
则由式\eqref{eq:transverse3}和式\eqref{eq:quadrupole_force}可得:
\begin{equation}
    \label{eq:quadrupole_force2}
    x'' = -\frac{qgx}{p_z} \approx -\frac{gx}{B\rho},
\end{equation}
其中${B\rho}$为磁刚度,即:
\begin{equation}
    {B\rho}=\frac{mv}{q} \approx \frac{mv_z}{q} \text{。}
\end{equation}

我们定义四极磁铁的聚焦强度$K$为:
\begin{equation}
    \label{eq:quadrupole_force3}
    K \equiv \frac{g}{B\rho} = \frac{q}{mv}                 \frac{\partial B_y}{\partial x}
                             = \frac{q}{\gamma m_0 \beta c} \frac{\partial B_y}{\partial x},
\end{equation}
则粒子的横向运动方程可以写为:
\begin{equation}
    \label{eq:Mathier_Hill_x}
    x'' + Kx =0 \text{。}
\end{equation}
在不同的加速器元件中,$K$会发生变化,而且在水平方向和垂直方向可能不同,因此一个更普遍的描述单粒子横向运动的方程可以表示为:
\begin{equation}
    \label{eq:Mathier_Hill}
    \left\{
    \begin{aligned}
        x'' + K_x(s)x &= 0\text{,} \\
        y'' + K_y(s)y &= 0\text{。}
    \end{aligned}
    \right.
\end{equation}
上式也被称为希尔方程,以X方向为例,其解可以表示为\cite{abramowitz1964handbook}:
\begin{equation}
    \label{eq:Mathier_Hill_solver}
    x(s)=\left\{
    \begin{aligned}
        &a\cos (\sqrt{K}s+b) ,& K>0\text{,} \\
        &as+b                ,& K=0\text{,} \\
        &a\cosh(\sqrt{-K}s+b),& K<0\text{。} \\
    \end{aligned}
    \right.
\end{equation}
当$K=0$时,表示聚焦强度为0,即粒子在漂移节中运动;当$K>0$时,表示聚焦作用;当$K<0$时,表示散焦作用。

由式\eqref{eq:Mathier_Hill_solver}可知,~$x$~和~$x'$~都是连续的,而积分常数~$a$~和~$b$~则取决于初始条件~$x(0)$~和~$x'(0)$。
记$\vec{x}(s)=[x(s),x'(s)]$为粒子运动的状态矢量,则方程\eqref{eq:Mathier_Hill}的解可以表示为矩阵形式:
\begin{equation}
    \label{eq:Mathier_Hill_solver_matrix}
    \begin{bmatrix}
      x   \\
      x'  \\
    \end{bmatrix}
    =
    \mathbf{M}
    \begin{bmatrix}
      x_0   \\
      x_0'  \\
    \end{bmatrix}
    ,
\end{equation}
其中$\mathbf{M}$为传输矩阵:
\begin{equation}
    \label{eq:Mathier_Hill_tranfermap}
    \mathbf{M}=
    \begin{bmatrix}
      r_{11} & r_{12}  \\
      r_{21} & r_{22}  \\
    \end{bmatrix}
\end{equation}
对于聚焦强度$K$为常数的结构,我们可以得到传输矩阵的一般形式:
\begin{equation}
    \label{eq:Mathier_Hill_tranfermap_K}
    \mathbf{M}=\left\{
    \begin{aligned}
        &\begin{bmatrix}
          \cos \sqrt{K}L                & \frac{1}{\sqrt{K}} \sin \sqrt{K}L  \\
          -L{\sqrt{K}} \sin \sqrt{K}L    & \cos \sqrt{K}L   \\
        \end{bmatrix}
        ,& K>0:\text{聚焦},  \\
        &\begin{bmatrix}
          1 & L  \\
          0 & 1  \\
        \end{bmatrix}
        ,& K=0:\text{漂移节}, \\
        &\begin{bmatrix}
          \cosh \sqrt{|K|}L                & \frac{1}{\sqrt{|K|}} \sinh \sqrt{|K|}L  \\
          L{\sqrt{|K|}} \sinh \sqrt{|K|}L    & \cosh \sqrt{|K|}L   \\
        \end{bmatrix}
        ,& K<0:\text{散焦}, \\
    \end{aligned}
    \right.
\end{equation}
其中$L$为初位置和末位置之间的距离。

在薄透镜近似下,即$L \rightarrow 0$情况下,传输矩阵可以简化为:
\begin{equation}
    \label{eq:Mathier_Hill_tranfermap_thinlen}
    \mathbf{M}=\left\{
    \begin{aligned}
        &\begin{bmatrix}
          1              & 0  \\
         - \frac{1}{f}   & 1  \\
        \end{bmatrix}
        ,& K>0:\text{聚焦},  \\
        &\begin{bmatrix}
          1 & L  \\
          0 & 1  \\
        \end{bmatrix}
        ,& K=0:\text{漂移节}, \\
        &\begin{bmatrix}
          1              & 0  \\
          \frac{1}{f}    & 1   \\
        \end{bmatrix}
        ,& K<0:\text{散焦}, \\
    \end{aligned}
    \right.
\end{equation}
式中$f = \lim \limits_{L \rightarrow 0} \frac{1}{\sqrt{K}L}$定义为透镜焦距。

传输矩阵更一般的表达为:
\begin{equation}
    \label{eq:Mathier_Hill_tranfermap2}
    \mathbf{M}=
    \begin{bmatrix}
      \cos \Phi + \alpha \sin \Phi    & \beta \sin \Phi  \\
      -\gamma \sin \Phi  & \cos \Phi - \alpha \sin \Phi  \\
    \end{bmatrix}
    =
    \textbf{I} \cos \Phi + \textbf{J} \sin \Phi ,
\end{equation}
其中$\alpha$、$\beta$、$\gamma$就是加速器中常用的Twiss参数,
并且$\beta \gamma = 1+{\alpha}^2$。
$\Phi$为相移,
$\textbf{I}$为单位矩阵,而$\textbf{J}$形式如下:
\begin{equation}
    \label{eq:Mathier_Hill_tranfermap3}
    \textbf{J}=
    \begin{bmatrix}
      \alpha    & \beta  \\
      -\gamma   & -\alpha \\
    \end{bmatrix} \text{。}
\end{equation}

\section{束流纵向运动}
加速器一般使用加速腔对带电粒子进行加速\cite{wang1986protronLinac,yao1986elecLinac,wangler1998principles}。
带电粒子要被加速,必须满足同步加速条件。
符合加速腔设计相位的理想粒子为同步粒子(参考粒子),而与设计相位存在偏差的粒子为非同步粒子。
经过一个加速单元后,同步粒子的能量增益$\delta W_s$和非同步粒子的能量增益$\delta W$分别为:
\begin{equation}
    \label{eq:longitudinal1}
    \begin{aligned}
        \delta W_s &= e E_0 T L_c \cos {\varphi}_s \text{,} \\
        \delta W   &= e E_0 T L_c \cos {\varphi}   \text{。}
    \end{aligned}
\end{equation}
其中,$e$为带电粒子电荷,$L_c$和$E_0$分别为加速腔的长度和加速梯度,
$T$为渡越时间因子,与加速腔的结构有关,
而${\varphi}_s$和${\varphi}$分别为同步粒子和非同步粒子的加速相位。于是有:
\begin{equation}
    \label{eq:longitudinal2}
        \frac{d\Delta W}{dz} = e E_0 T (\cos {\varphi} - \cos {\varphi}_s),
\end{equation}
其中
\begin{equation}
    \label{eq:longitudinal2_cont}
        \Delta W = m_0 c^2 {\gamma}_s^3 {\beta}_s \Delta \beta, \qquad \Delta \beta = \beta - {\beta}_s \text{。}
\end{equation}
其中的$\beta$和$\gamma$分别为粒子的相对论速度和相对论因子。
另外,相位变化也和速度变化有关:
\begin{equation}
    \label{eq:longitudinal3}
        \Delta \varphi = \varphi - {\varphi}_s = -\frac{z-z_s}{\beta _s \gamma} 2\pi
        \quad \rightarrow \quad
        \Delta \beta = - \frac{\beta_s^2 \gamma}{2\pi} \frac{d \Delta \varphi}{dz} \text{。}
\end{equation}
由式\eqref{eq:longitudinal2}和式\eqref{eq:longitudinal3}可得:
\begin{equation}
    \label{eq:longitudinal4}
        \Delta \gamma = \gamma - \gamma _s
                = \frac{\Delta W}{m_0 c^2}
                = -\frac{\lambda}{2\pi} \beta_s^3 \gamma_s^3 \frac{d\Delta \varphi}{dz} \text{。}
\end{equation}
联立上式与式\eqref{eq:longitudinal2},即可得到纵向运动方程:
\begin{equation}
    \label{eq:longitudinal_equation}
        \frac{1}{\beta_s^3 \gamma_s^3} \frac{d}{dz}\left(\beta_s^3 \gamma_s^3 \frac{d\Delta \varphi}{dz}\right)
        +\frac{2\pi e E_0 T}{m_0 c^2 \beta_s^3 \gamma_s^3 \lambda} (\cos {\varphi} - \cos {\varphi}_s)
        =0 \text{。}
\end{equation}
展开得到:
\begin{equation}
    \label{eq:longitudinal_equation2}
        \frac{d^2 \Delta \varphi}{dz^2}
        +\frac{3}{\beta_s \gamma_s} \frac{d\beta_s \gamma_s}{dz} \frac{d\Delta \varphi}{dz}
        +\frac{2\pi e E_0 T}{m_0 c^2 \beta_s^3 \gamma_s^3 \lambda} (\cos {\varphi} - \cos {\varphi}_s)
        =0 \text{。}
\end{equation}
纵向运动方程描述了粒子在($\Delta W , \Delta \varphi$)相空间中的运动。假设加速梯度足够小,我们可以忽略阻尼项$\frac{d\beta_s \gamma_s}{dz}$,则纵向运动方程可以简化为:
\begin{equation}
    \label{eq:longitudinal_equation2_concise}
        \frac{d^2 \Delta \varphi}{dz^2}
        +\frac{2\pi e E_0 T}{m_0 c^2 \beta_s^3 \gamma_s^3 \lambda} (\cos {\varphi} - \cos {\varphi}_s)
        =0 \text{。}
\end{equation}

\begin{figure}[!htb]
    \centering
    \includegraphics[width=0.5\textwidth]{Img/longitudinal.pdf}
    \caption{加速腔电场、势阱、相空间轨迹和相稳定区示意图}
    \label{fig:longitudinal}
\end{figure}

加速腔梯度和相位的关系见图\eqref{fig:longitudinal}最上方曲线\cite{wangler1998principles}。
当同步相位处于$(-\frac{\pi}{2},0)$之间时,粒子纵向相空间存在势阱,
相稳定区的边界为“鱼形”,如图\eqref{fig:longitudinal}中图所示。
势阱中的粒子围绕同步粒子$(\varphi_s,W_s)$作振荡,能够被稳定加速。
其稳定区的范围为:
\begin{equation}
    \label{eq:longitudinal_phi}
        \varphi_2 < \varphi < -\varphi_s \text{。}
\end{equation}
对于小角度振荡,$\varphi_2 \approx 2\varphi_s$,因此稳定区的宽度约为$3|\varphi_s|$。
当$|\varphi_s|$增加时,稳定区会增大,但是由于同步相位为负,所以总能量增益会减小。

前面的分析中忽略了阻尼项,但在粒子能量较低时阻尼项不可忽略。当考虑了阻尼项之后,相空间的轨迹也有所变化。
对于低能量的粒子,当经过加速后,同步粒子的速度和能量会有较大变化,相稳定区将有所增大,
因而加速器的接收范围有所增加,相稳定区域的边界也由原来的“鱼形”变成“螺旋线形”形,如图\eqref{fig:longitudinal1}所示\cite{wangler1998principles}。

\begin{figure}[!htb]
    \centering
    \begin{subfigure}[b]{0.48\textwidth}
        \includegraphics[width=\textwidth]{Img/longitudinal1.pdf}
    \end{subfigure}
    \begin{subfigure}[b]{0.45\textwidth}
        \includegraphics[width=\textwidth]{Img/longitudinal2.pdf}
    \end{subfigure}
    \caption{考虑阻尼项后的纵向相空间稳定区示意图}
    \label{fig:longitudinal1}
\end{figure}


\section{空间电荷效应}
\label{section:spaceChargeIntro}
如小节\eqref{section:background}中介绍,空间电荷效应的来源为束团内的粒子之间的库伦相互作用。
随着加速器流强的不断提高,束团内粒子的相互作用与外场对粒子的作用相比已经不可忽略,
甚至在极端流强下,空间电荷对粒子的运动起到了主导作用。

在一个多粒子的束团中,空间电荷效应产生的力可以分为两个方面:
一方面来源于长程的电磁相互作用,即束团中所有粒子在空间中产生了一个近似光滑的电磁场,每一个处于这个电磁场中的粒子都受到作用;
另一方便来源于短程的库伦碰撞作用。
通常来讲,考虑到加速器中的一个束团的粒子数目,相比于短程的库伦碰撞作用,长程的电磁相互作用占据了主导。
所以一般在空间电荷研究中,只考虑长程的相互作用,而忽略短程的碰撞作用。
短程的库伦碰撞引起的散射叫做束内散射(Intra-Beam Scattering, IBS),其不在本文讨论范围内。
在本文中,空间电荷效应指束团内部的长程电磁相互作用。

空间电荷作用与粒子在空间中的分布有关,而粒子的分布又是加速器元件产生的外场和束团自身产生的内场共同作用的结果。
解析的计算需要联立求解牛顿方程和麦克斯韦方程组(式\eqref{eq:Newton}和\eqref{eq:Maxwell}),其精确解只能使用数值的方法来得到。

在早期计算机性能不足时,人们使用很多方法对空间电荷效应进行了近似估计。
比如空间电荷线性近似方法,在这种方法中,我们首先假定束团的分布形态,然后使用连续电荷分布所产生的场表达粒子间的相互作用。
例如在三维情况下,如果粒子在实空间中的分布为一个均匀的椭球,则粒子在空间中任意位置的所受到的电场可以表示为\cite{lv2004beamoptic}:
\begin{equation}
    \label{eq:SpaceCharge3D}
    \begin{aligned}
    E_x &= \frac{3ITx}{4 \pi {\varepsilon}_{0} {\gamma}^2 XYZ}{\mu}_x \text{,} \\
    E_y &= \frac{3ITy}{4 \pi {\varepsilon}_{0} {\gamma}^2 XYZ}{\mu}_y \text{,} \\
    E_z &= \frac{3ITz}{4 \pi {\varepsilon}_{0} {\gamma}^2 XYZ}{\mu}_z \text{。}
    \end{aligned}
\end{equation}
其中$I$为束流流强,$T$为相邻束团的时间间隔,
$X$,$Y$,$Z$分别是束流在三个方向上的尺寸,$x$,$y$,$z$是粒子的位置,
${\mu}_x$,${\mu}_y$,${\mu}_z$是束团的形状因子:
\begin{equation}
    \label{eq:SpaceCharge3D_mu}
    \begin{aligned}
    {\mu}_x &= \frac{XYZ\gamma}{2}   \int_0^{\infty}
    \frac{d\xi}{(X^2+\xi)            \sqrt{(X^2+\xi)(Y^2+\xi)(Z^2 {\gamma}^2+\xi)}} \text{,}  \\
    {\mu}_y &= \frac{XYZ\gamma}{2}   \int_0^{\infty}
    \frac{d\xi}{(Y^2+\xi)            \sqrt{(X^2+\xi)(Y^2+\xi)(Z^2 {\gamma}^2+\xi)}} \text{,} \\
    {\mu}_z &= \frac{XYZ\gamma}{2}   \int_0^{\infty}
    \frac{d\xi}{(Z^2 {\gamma}^2+\xi) \sqrt{(X^2+\xi)(Y^2+\xi)(Z^2 {\gamma}^2+\xi)}} \text{。} \\
    \end{aligned}
\end{equation}

但是,由于线性近似方法对束团分布的硬性假设,这种方法得到的解是非自洽的。
而且这种方法只能得到线性空间电荷力的表达,几乎不能与实验相符。

计算机性能得到发展之后,人们更多的使用数值方法对空间电荷效应进行研究。
一种最直接的数值求解空间电荷效应的方法是叠加计算,即对一个粒子逐个计算与其他粒子的相互作用力,然后将其作用力叠加起来,构成该粒子的总空间电荷力。
这一种方法的计算复杂度为$O(N_p^2)$,$N_p$ 是粒子数目。
由于其运算量与粒子数目平方成正比,这种算法在粒子数目较大时的计算开销很大,在实际研究中并不常用。

为了降低运算量,人们采取另一种求解空间电荷效应的方法,质点网格法(也叫粒子云算法,英文简称PIC算法)。
其原理为通过对空间进行网格划分,先将粒子权重分配到网格上,再在网格上求解泊松方程,
得到空间网格上的电势分布后,通过差分得到网格上的电场,再通过反向插值将电场作用到单个粒子上。
通过使用网格,质点网格法的计算复杂度由原先直接粒子-粒子计算的$O(N_p^2)$ 降低到了$O(\alpha N_p + \beta N_{cells}\log{N_{cells}})$,
其中 $N_p$ 是粒子数,而$N_{cells}$ 是网格点数目,而$\alpha$和$\beta$为算法有关的常数。
因为PIC算法能够有效地降低运算量,所以绝大多数束流模拟程序使用PIC算法来求解空间电荷力。
关于PIC算法的详细介绍将会在第\eqref{section:PIC_algorithm}节中展开。

PIC算法需要将粒子权重到网格上,这样不可避免的会带来网格热噪声。另外目前人们对于PIC算法是否可以保证辛条件仍然有较大的争议。
如果不能保证辛条件,那么一些数值算法带来的非物理的效应就会被引入到模拟中。
最近,无网格保辛多粒子追踪算法(Symplectic算法)被引入到加速器研究和模拟,被用作长距离模拟中空间电荷求解器 \cite{symplectic_ji2017}。
Symplectic算法并不利用网格,而是利用高阶分解来求解空间电荷效应,避免了网格噪声带来的影响。
然而,Symplectic算法虽然能够保证辛条件,但其计算量要大得多,计算花费的时间比PIC算法要高两到三个数量级。
幸运的是,无网格算法很适合并行加速运算,有很好的可扩展性。
我们将在后文\eqref{section:symplectic_theory}节中对Symplectic算法的基本原理进行介绍。

\section{国内外强流加速器简介}
在早期,粒子加速器主要服务于高能物理研究,追求更高的能量,加速器发展也主要是朝着更高能量的方向发展。
近年来,强流逐渐成为了另外一个重要的研究方向,
其中一方面是因为某些物理现象为了提供足够高统计度,需要大量的事例累积;
另一方面是因为某些极为稀少的事例只有在流强较大的加速器中才能观察到。
图\eqref{fig:proton_ring}展示了正在运行的、建造中的和计划中的高流强高功率质子同步加速器\cite{tang2011proton},三条橙色虚线分别代表0.1MW、1MW、10MW的束流功率。
可以看出,受到核物理、高能物理、散裂中子源、加速器驱动的次临界核能系统等应用的驱动,强流质子加速器得到了极大的发展。
\begin{figure}[!htb]
    \centering
    \includegraphics[width=0.7\textwidth]{Img/protonAcc.pdf}
    \caption{高流强高功率质子同步加速器汇总}
    \label{fig:proton_ring}
\end{figure}

表\eqref{tab:proton_linac}和\eqref{tab:proton_ring}是目前正在运行、建造和计划中的高功率质子直线加速器和同步加速器的主要参数\cite{tang2011proton}。
目前运行的机器主要在1mA量级,功率在0.2-1MW之间。而在建和提出的加速器的功率基本在5MW附近。

\begin{table}[!htb]
  \centering
  \caption{国内外强流质子直线加速器参数表}
  \begin{tabular}{|>{\small}l|c|c|c|c|c|c|c|}
  %\begin{tabular}{>{\small}l|c|p{0.1\linewidth}|p{0.1\linewidth}|p{0.1\linewidth}|p{0.1\linewidth}|p{0.1\linewidth}|c|c|c}
    \hline
                &能量      &脉冲束长   &重复频率        &占空比       &$I_{b}$   &$I_{ave}$ &$P_{ave}$ \\
                &(GeV)  &(ms)  &(Hz)        &(\%)     &(mA)      &(mA)      &(MW)      \\
%                &       &    &         &     &          &          &          \\
    \hline
    LANSCE      &0.8   &0.625    &100/20    &6.2/1.2  &16/9.1    &1.0/0.1   &0.8/0.08  \\
    FNAL        &0.4   &0.05     &15        &0.04     &35        &0.014     &0.007     \\
    SNS         &1.0   &1.0      &60        &6.0      &38        &1.4       &1.4       \\
    J-PARC(1)   &0.18  &0.5      &50/25     &2.5      &50        &0.7       &0.28/0.14 \\
    J-PARC(2)   &0.6   &0.5      &25        &1.25     &50        &0.35      &0.21      \\
    CERN SPL    &2.2   &2.8      &50        &14       &22        &1.8       &4.0       \\
    ESS SP      &1.33  &1.2      &50        &6.0      &114       &3.75      &5.0       \\
    ESS LP      &1.33  &2.0/2.5  &16.67     &4.2      &114/90    &3.75      &5.0       \\
    ESS-S       &2.5   &2.86     &14        &4.0      &50        &2         &5.0       \\
    Project-X   &3/8   &Chopped  &-         &10/2.5   &10        &1/0.25    &3/2       \\
    TRASCO      &$\geqslant 1.0$ &CW  &-      &100      &30        &30        &$\geqslant 30$ \\
    IFMIF       &0.04  &CW       &-         &100      &2*125     &2*125     &10        \\
    C-ADS       &0.15  &CW       &-         &100      &10        &10        &1.5       \\
    \hline
  \end{tabular}
  \label{tab:proton_linac}
\end{table}

\begin{table}[!htb]
  \centering
  \caption{国内外强流质子环形加速器参数表}
  \begin{tabular}{|>{\small}l|c|c|c|c|c|c|}
  %\begin{tabular}{>{\small}l|c|p{0.1\linewidth}|p{0.1\linewidth}|p{0.1\linewidth}|p{0.1\linewidth}|p{0.1\linewidth}|c|c|c}
    \hline
                &能量       &注入能量   &重复频率      &累积粒子数    &$I_{ave}$ &$P_{ave}$ \\
                &(GeV)  &(GeV)  &(Hz)       &($10^{13}$)  &(mA)      &(MW)      \\
%                &       &        &          & &       &          \\
    \hline
    IPNS        &0.45   &0.05    &30        &0.3      &0.167     &0.0075    \\
    ISIS        &0.8    &0.07    &50        &3.75     &0.3       &0.24      \\
    PSR-I       &0.8    &0.8     &20        &3.1      &0.1       &0.08      \\
    PSR-II      &0.8    &0.8     &30        &4.1      &0.2       &0.16      \\
    SNS         &1.0    &1.0     &60        &14.6     &1.4       &1.4       \\
    J-PARC/RCS  &3.0    &0.4     &25        &8.3      &0.333     &1.0       \\
    J-PARC/MR   &50     &3.0     &0.3       &33.2     &0.015     &0.75      \\
    FNAL/Booster&8      &0.4     &15        &7.5      &0.014     &0.12      \\
%    ESS         &1.33   &1.33    &50        &2*23.4   &2*1.875   &5.0       \\
%    AUSTRON     &1.6    &0.13    &50        &3.9      &0.313     &0.5       \\
    CSNS-I      &1.6    &0.8     &25        &1.56     &0.065     &0.1       \\
    CSNS-II     &1.6    &0.25    &25        &7.8      &0.325     &0.5       \\
    ISNS        &1.0    &0.1     &25        &2.4      &0.1       &0.1       \\
    Project-X/MI&120    &8       &0.4       &16       &0.0167    &2         \\
    \hline
  \end{tabular}
  \label{tab:proton_ring}
\end{table}




