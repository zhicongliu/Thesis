
\chapter{强流束流动力学}
\label{chap:AccBasic}

\section{基本理论}
一个电荷为$e$的带电粒子,在电场强度为$\vec{E}$、磁感应强度为$\vec{B}$的电磁场中运动,其运动方程为:
\begin{equation}
    \label{eq:Newton}
    \frac{d \vec{p}}{dt}=e(\vec{E}+\vec{v} \times \vec{B})
\end{equation}
其中,$\vec{p}=\gamma m \vec{v}$为粒子的动量,
$\vec{v}$是带电粒子的速度,
$m$为粒子的固有质量,
$\gamma$为洛伦兹因子:
\begin{equation}
    \label{eq:Lorentz}
    \gamma = \frac{1}{\sqrt{1-\frac{v^2}{c^2}}}, \qquad v = \sqrt{\vec{v} \cdot \vec{v}},
\end{equation}
$c$为光速。

电磁场满足麦克斯韦方程组。在真空中,麦克斯韦方程可以表达为:
\begin{equation}
    \label{eq:Maxwell}
    \begin{aligned}
    \nabla \cdot \vec{E}  &= \frac{\rho}{{\varepsilon}_{0}}, \\
    \nabla \cdot \vec{B}  &= 0, \\
    \nabla \times \vec{E} &= -\frac{\partial \vec{B}}{\partial {t}}, \\
    \nabla \times \vec{B} &= {\mu}_{0}\left(\vec{J}+\varepsilon \frac{\partial \vec{E}}{\partial {t}}\right).
    \end{aligned}
\end{equation}
其中,$\rho$为电荷密度,$\varepsilon _0$为真空介电常数,${\mu}_{0}$为真空磁导率,$\vec{J}$为电流密度。

\section{束流横向运动}

我们定义束流横向运动的与纵向运动的夹角为$x'$:
\begin{equation}
    \label{eq:transverse1}
    x'=\frac{dx}{dz}=\frac{p_x}{p_z}
\end{equation}
则
\begin{equation}
    \label{eq:transverse2}
    F_x =\frac{dp_x}{dt}=\frac{dp_z}{dt}\frac{dz}{dt}=v_z \frac{d(p_z x')}{dz}=v_z(p_z x''+p_z'x')
\end{equation}
假设束流没有被加速,即$p_z'=0$,则粒子的横向运动方程为:
\begin{equation}
    \label{eq:transverse3}
    x'' = \frac{F_x}{p_z v_z}
\end{equation}

现代加速器中一般使用二极磁铁来对束流进行偏转,使用四级磁铁来对束流进行横向聚焦。
四极磁铁的磁场分布和粒子受力如图\ref{fig:quadrupole}所示\cite{qinq2011ring}。
可以看出,四级铁产生的力在一个方向上聚焦,而在另一个方向上散焦。
粒子所受到的力与粒子的位置成正比,以X方向为例,受力如式\ref{eq:quadrupole_force}所示。
\begin{figure}[!htb]
    \centering
    \includegraphics[width=0.99\textwidth]{Img/quadrupole.pdf}
    \caption{四极磁铁中磁场及粒子受力(带负电荷的粒子垂直于纸面向里运动)}
    \label{fig:quadrupole}
\end{figure}
\begin{equation}
    \label{eq:quadrupole_force}
    F_x = -q {v}_{z} {B}_{y} = -q {v}_{z} g x, \qquad g=\frac{\partial B_y}{\partial x}
\end{equation}
则由式\ref{eq:transverse3}和式\ref{eq:quadrupole_force}可得:
\begin{equation}
    \label{eq:quadrupole_force2}
    x'' = -\frac{qgx}{p_z} \approx -\frac{gx}{B\rho}
\end{equation}
其中${B\rho}$为磁刚度:
\begin{equation}
    {B\rho}=\frac{mv}{q}\approx{mv_z}{q}
\end{equation}

我们定义四极磁铁的聚焦强度$K$为:
\begin{equation}
    \label{eq:quadrupole_force3}
    K \equiv \frac{g}{B\rho} = \frac{q}{mv}                 \frac{\partial B_y}{\partial x} 
                             = \frac{q}{\gamma m_0 \beta c} \frac{\partial B_y}{\partial x} 
\end{equation}
则粒子的横向运动方程可以写为:
\begin{equation}
    \label{eq:Mathier_Hill_x}
    x'' + Kx =0
\end{equation}
类似的,Y方向的运动方程为:
\begin{equation}
    \label{eq:Mathier_Hill_y}
    y'' - Ky =0
\end{equation}

以上两式\ref{eq:Mathier_Hill_x}和\ref{eq:Mathier_Hill_y}描述了粒子的横向运动,被称为希尔方程。

\section{束流纵向运动}

\section{空间电荷效应}
如小节\ref{section:background}中介绍,空间电荷效应的来源为束团本身。
随着加速器流强的不断提高,束团自身的相互作用与外场对粒子的作用相比已经不可忽略,
甚至在极端流强下,空间电荷对粒子的运动起到了主导作用。

在一个多粒子的束团中,空间电荷效应产生的力可以分为两个方面:
一方面来源于长程的电磁相互作用,即束团中所有粒子在空间中产生了一个近似光滑的电磁场,每一个处于这个电磁场中的粒子都受到作用;
另一方便来源于短程的库伦碰撞作用。
通常来讲,考虑到加速器中的一个束团的粒子数目,相比于短程的库伦碰撞作用,长程的电磁相互作用占据了主导。
所以一般在空间电荷研究中,只考虑长程的相互作用,而忽略短程的碰撞作用。
短程的库伦碰撞引起的散射叫做束内散射(IBS,Intra-Beam Scattering),其不在本文讨论范围内。
在本文中,空间电荷效应指束团内部的长程电磁相互作用。

空间电荷作用与粒子在空间中的分布有关,而粒子的分布又是加速器元件产生的外场和束团自身产生的内场共同作用的结果,
其组成比较复杂,解析计算的话需要联立求解牛顿方程和麦克斯韦方程组(式\ref{eq:Newton}和\ref{eq:Maxwell})。
而对于这组方程,我们目前并不能给出解析解,因此其精确解必须使用数值的方法来得到。

在早期计算机性能不足时,人们使用很多方法对空间电荷效应进行了近似估计。
其中线性近似是最早发展起来的一种方法\cite{lv2004beamoptic}。

At HERE

计算机发展之后,人们使用数值方法对空间电荷效应进行研究。
一种最直接的数值求解空间电荷效应的方法是叠加计算,即对一个粒子逐个计算与其他粒子的相互作用力,然后将其作用力叠加起来,构成该粒子的总空间电荷力。
这一种方法的计算复杂度为$O(N_p^2)$,$N_p$ 是粒子数目。
由于其运算量与粒子数目平方成正比,这种算法在粒子数目较大时的计算开销很大。

另一种求解空间电荷效应的方法为质点网格法(PIC方法),
原理为通过对空间进行网格划分,先将粒子权重到网格上,再在网格上求解泊松方程,
得到空间网格上的电势分布后,再将其作用到单个粒子上。
通过使用网格,PIC的计算复杂度由直接计算的$O(N_p^2)$$N_p^2$ 降低到了$O(\alpha N_p + \beta N_{cells}\log{N_{cells}})$,
其中 $N_p$ 是粒子数,而$N_{cells}$ 是网格点数目,而$\alpha$和$\beta$为算法有关的常数。
由于PIC算法能够有效地降低运算量,绝大多数加速器束流模拟程序使用PIC算法来求解空间电荷力。
关于PIC算法的详细介绍将会在第\ref{section:PIC_algorithm}节中展开。

最近,无网格保辛多粒子追踪模型被引入到加速器研究和模拟中,作为在长距离模拟中空间电荷求解器 \cite{symplectic_ji2017}。
这是因为PIC算法需要将粒子权重到网格上,不可避免的会带来网格热噪声,因此PIC算法是否可以保障辛条件(symplectic)在目前仍然有较大的争议。
如果不能报障辛条件,那么计算就会被引入一些数值算法带来的非物理的效应。
Symplectic算法并不利用网格,而是利用高阶分解来求解空间电荷效应。
然而,Symplectic算法虽然能够保证辛条件,但其计算量要大得多,计算花费的时间比PIC算法要高两到三个数量级。
幸运的是,无网格算法很适合并行加速运算,有很好的可扩展性。
我们将在后文\ref{section:symplectic_theory}节中对保辛算法的基本原理进行介绍。


\section{国内外强流加速器简介}

\section{小结}