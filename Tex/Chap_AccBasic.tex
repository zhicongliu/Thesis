
\chapter{强流束流动力学}
\label{chap:AccBasic}

\section{基本理论}
一个电荷为$e$的带电粒子,在电场强度为$\vec{E}$、磁感应强度为$\vec{B}$的电磁场中运动,其运动方程为:
\begin{equation}
    \label{eq:Newton}
    \frac{d \vec{p}}{dt}=e(\vec{E}+\vec{v} \times \vec{B})
\end{equation}
其中,$\vec{p}=\gamma m \vec{v}$为粒子的动量,
$\vec{v}$是带电粒子的速度,
$m$为粒子的固有质量,
$\gamma$为洛伦兹因子:
\begin{equation}
    \label{eq:Lorentz}
    \gamma = \frac{1}{\sqrt{1-\frac{v^2}{c^2}}}, \qquad v = \sqrt{\vec{v} \cdot \vec{v}},
\end{equation}
$c$为光速。

电磁场满足麦克斯韦方程组。在真空中,麦克斯韦方程可以表达为:
\begin{equation}
    \label{eq:Maxwell}
    \begin{aligned}
    \nabla \cdot \vec{E}  &= \frac{\rho}{{\varepsilon}_{0}}, \\
    \nabla \cdot \vec{B}  &= 0, \\
    \nabla \times \vec{E} &= -\frac{\partial \vec{B}}{\partial {t}}, \\
    \nabla \times \vec{B} &= {\mu}_{0}\left(\vec{J}+\varepsilon \frac{\partial \vec{E}}{\partial {t}}\right).
    \end{aligned}
\end{equation}
其中,$\rho$为电荷密度,$\varepsilon _0$为真空介电常数,${\mu}_{0}$为真空磁导率,$\vec{J}$为电流密度。

\section{束流横向运动}

我们定义束流横向运动的与纵向运动的夹角为$x'$:
\begin{equation}
    \label{eq:transverse1}
    x'=\frac{dx}{dz}=\frac{p_x}{p_z}
\end{equation}
则
\begin{equation}
    \label{eq:transverse2}
    F_x =\frac{dp_x}{dt}=\frac{dp_z}{dt}\frac{dz}{dt}=v_z \frac{d(p_z x')}{dz}=v_z(p_z x''+p_z'x')
\end{equation}
假设束流没有被加速,即$p_z'=0$,则粒子的横向运动方程为:
\begin{equation}
    \label{eq:transverse3}
    x'' = \frac{F_x}{p_z v_z}
\end{equation}

现代加速器中一般使用二极磁铁来对束流进行偏转,使用四级磁铁来对束流进行横向聚焦。
四极磁铁的磁场分布和粒子受力如图\ref{fig:quadrupole}所示\cite{qinq2011ring},其中带负电荷的粒子垂直于纸面向里运动。
可以看出,四级铁产生的力在一个方向上聚焦,而在另一个方向上散焦。
粒子所受到的力与粒子的位置成正比,以X方向为例,受力如式\ref{eq:quadrupole_force}所示。
\begin{figure}[!htb]
    \centering
    \includegraphics[width=0.99\textwidth]{Img/quadrupole.pdf}
    \caption{四极磁铁中磁场及粒子受力}
    \label{fig:quadrupole}
\end{figure}
\begin{equation}
    \label{eq:quadrupole_force}
    F_x = -q {v}_{z} {B}_{y} = -q {v}_{z} g x, \qquad g=\frac{\partial B_y}{\partial x}
\end{equation}
则由式\ref{eq:transverse3}和式\ref{eq:quadrupole_force}可得:
\begin{equation}
    \label{eq:quadrupole_force2}
    x'' = -\frac{qgx}{p_z} \approx -\frac{gx}{B\rho}
\end{equation}
其中${B\rho}$为磁刚度:
\begin{equation}
    {B\rho}=\frac{mv}{q}\approx{mv_z}{q}
\end{equation}

我们定义四极磁铁的聚焦强度$K$为:
\begin{equation}
    \label{eq:quadrupole_force3}
    K \equiv \frac{g}{B\rho} = \frac{q}{mv}                 \frac{\partial B_y}{\partial x}
                             = \frac{q}{\gamma m_0 \beta c} \frac{\partial B_y}{\partial x}
\end{equation}
则粒子的横向运动方程可以写为:
\begin{equation}
    \label{eq:Mathier_Hill_x}
    x'' + Kx =0
\end{equation}

在不同的加速器元件中,$K$会发生变化,因此一个更普遍的描述单粒子横向运动的方程可以表示为:
\begin{equation}
    \label{eq:Mathier_Hill}
    \begin{aligned}
        x'' + K_x(s)x &= 0 \\
        y'' + K_y(s)y &= 0
    \end{aligned}
\end{equation}
上式也被称为希尔方程,以X方向为例,其解可以表示为\cite{abramowitz1964handbook}:
\begin{equation}
    \label{eq:Mathier_Hill_solver}
    x(s)=\left\{
    \begin{aligned}
        &a\cos (\sqrt{K}s+b) ,& K>0 \\
        &as+b                ,& K=0 \\
        &a\cosh(\sqrt{-K}s+b),& K<0 \\
    \end{aligned}
    \right.
\end{equation}
当$K=0$时,表示聚焦强度为0,即粒子在漂移节中运动;当$K>0$时,表示聚焦作用;当$K<0$时,表示散焦作用。

由\ref{eq:Mathier_Hill_solver}可知,$x$和$x'$都是连续的,而积分常数$a$和$b$则取决于初始条件$x(0)$和$x'(0)$。
记$\vec{x}(s)=[x(s),x'(s)]$为粒子运动的状态矢量,则方程\ref{eq:Mathier_Hill}的解可以表示为矩阵形式:
\begin{equation}
    \label{eq:Mathier_Hill_solver_matrix}
    \begin{bmatrix}
      x   \\
      x'  \\
    \end{bmatrix}
    =
    M
    \begin{bmatrix}
      x_0   \\
      x_0'  \\
    \end{bmatrix}
\end{equation}
其中M叫做传输矩阵:
\begin{equation}
    \label{eq:Mathier_Hill_tranfermap}
    M=
    \begin{bmatrix}
      R_{11} & R_{12}  \\
      R_{21} & R_{22}  \\
    \end{bmatrix}
\end{equation}
对于聚焦强度K为常数的结构,我们可以得到传输矩阵的一般形式:
\begin{equation}
    \label{eq:Mathier_Hill_tranfermap_K}
    M=\left\{
    \begin{aligned}
        &\begin{bmatrix}
          \cos \sqrt{K}L                & \frac{1}{\sqrt{K}} \sin \sqrt{K}L  \\
          L{\sqrt{K}} \sin \sqrt{K}L    & \cos \sqrt{K}L   \\
        \end{bmatrix}
        ,& K>0:focusing  \\
        &\begin{bmatrix}
          1 & L  \\
          0 & 1  \\
        \end{bmatrix}
        ,& K=0:drift  \\
        &\begin{bmatrix}
          \cosh \sqrt{|K|}L                & \frac{1}{\sqrt{|K|}} \sinh \sqrt{|K|}L  \\
          L{\sqrt{|K|}} \sinh \sqrt{|K|}L    & \cosh \sqrt{|K|}L   \\
        \end{bmatrix}
        ,& K>0:focusing  \\
    \end{aligned}
    \right.
\end{equation}
其中L为初位置和末位置之间的距离。

在薄透镜近似下,即$L \rightarrow 0$情况下,传输矩阵可以简化为:
\begin{equation}
    \label{eq:Mathier_Hill_tranfermap_thinlen}
    M=\left\{
    \begin{aligned}
        &\begin{bmatrix}
          1              & 0  \\
         - \frac{1}{f}   & 1  \\
        \end{bmatrix}
        ,& K>0:focusing  \\
        &\begin{bmatrix}
          1 & L  \\
          0 & 1  \\
        \end{bmatrix}
        ,& K=0:drift  \\
        &\begin{bmatrix}
          1              & 0  \\
          \frac{1}{f}    & 1   \\
        \end{bmatrix}
        ,& K>0:focusing  \\
    \end{aligned}
    \right.
\end{equation}
式中$f=\lim_{L \rightarrow 0} \frac{1}{\sqrt{K}L}$为焦距。

传输矩阵更一般的表达为:
\begin{equation}
    \label{eq:Mathier_Hill_tranfermap2}
    M=
    \begin{bmatrix}
      \cos \Phi + \alpha \sin \Phi    & \beta \sin \Phi  \\
      -\gamma \sin \Phi  & \cos \Phi - \alpha \sin \Phi  \\
    \end{bmatrix}
    =
    \textbf{I} \cos \Phi + \textbf{J} \sin \Phi
\end{equation}
其中$\alpha$$\beta$$\gamma$就是加速器中常用的Twiss参数,
并且$\beta \gamma = 1+{\alpha}^2$。
$\Phi$为相移,
$\textbf{I}$为单位矩阵,而$\textbf{J}$形式如下:
\begin{equation}
    \label{eq:Mathier_Hill_tranfermap3}
    \textbf{J}=
    \begin{bmatrix}
      \alpha    & \beta  \\
      -\gamma   & \alpha \\
    \end{bmatrix}
\end{equation}


\section{束流纵向运动}
加速器一般使用各种加速腔对带电粒子进行加速。
带电粒子(参考粒子)要获取加速,必须满足各自相应的同步加速条件。
经过一个加速单元后,同步粒子和非同步粒子的能量增益分别为:
\begin{equation}
    \label{eq:longitudinal1}
    \begin{aligned}
        \delta W_s &= e E_0 T L_c \cos {\varphi}_s \\
        \delta W   &= e E_0 T L_c \cos {\varphi}
    \end{aligned}
\end{equation}
其中$e$为带电粒子电荷,$L_c$和$E_0$分别为加速腔的长度和加速梯度,
$T$为渡越时间因子,与加速腔的结构有关,
而${\varphi}_s$和${\varphi}$分别为同步粒子和非同步粒子的加速相位。于是有:
\begin{equation}
    \label{eq:longitudinal2}
        \frac{d\Delta W}{dz} = e E_0 T (\cos {\varphi} - \cos {\varphi}_s)
\end{equation}
其中
\begin{equation}
    \label{eq:longitudinal2_cont}
        \Delta W = m_0 c^2 {\gamma}_s^3 {\beta}_s \Delta \beta, \qquad \Delta \beta = \beta - {\beta}_s
\end{equation}

另外相位变化也和速度变化有关:
\begin{equation}
    \label{eq:longitudinal3}
        \Delta \varphi = \varphi - {\varphi}_s = -\frac{z-z_s}{\beta _s \gamma} 2\pi
        \quad \rightarrow \quad
        \Delta \beta = - \frac{\beta_s^2 \gamma}{2\pi} \frac{d \Delta \varphi}{dz}
\end{equation}

由式\ref{eq:longitudinal2}和式\ref{eq:longitudinal3}可得:
\begin{equation}
    \label{eq:longitudinal4}
        \Delta \gamma = \gamma - \gamma _s
                = \frac{\Delta W}{m_0 c^2}
                = -\frac{\lambda}{2\pi} \beta_s^3 \gamma_s^3 \frac{d\Delta \varphi}{dz}
\end{equation}

联立上式与式\ref{eq:longitudinal2},即可得到纵向运动方程:
\begin{equation}
    \label{eq:longitudinal_equation}
        \frac{1}{\beta_s^3 \gamma_s^3} \frac{d}{dz}\left(\beta_s^3 \gamma_s^3 \frac{d\Delta \varphi}{dz}\right)
        +\frac{2\pi e E_0 T}{m_0 c^2 \beta_s^3 \gamma_s^3 \lambda} (\cos {\varphi} - \cos {\varphi}_s)
        =0
\end{equation}
展开得到:
\begin{equation}
    \label{eq:longitudinal_equation2}
        \frac{d^2 \Delta \varphi}{dz^2}
        +\frac{3}{\beta_s \gamma_s} \frac{d\beta_s \gamma_s}{dz} \frac{d\Delta \varphi}{dz}
        +\frac{2\pi e E_0 T}{m_0 c^2 \beta_s^3 \gamma_s^3 \lambda} (\cos {\varphi} - \cos {\varphi}_s)
        =0
\end{equation}
纵向运动方程描述了粒子在($\Delta W , \Delta \varphi$)相空间中的运动。假设加速梯度足够小,我们可以忽略阻尼项$\frac{d\beta_s \gamma_s}{dz}$,则纵向运动方程可以简化为:
\begin{equation}
    \label{eq:longitudinal_equation2_concise}
        \frac{d^2 \Delta \varphi}{dz^2}
        +\frac{2\pi e E_0 T}{m_0 c^2 \beta_s^3 \gamma_s^3 \lambda} (\cos {\varphi} - \cos {\varphi}_s)
        =0
\end{equation}

\begin{figure}[!htb]
    \centering
    \includegraphics[width=0.5\textwidth]{Img/longitudinal.pdf}
    \caption{加速腔电场、势阱、相空间轨迹和相稳定区}
    \label{fig:longitudinal}
\end{figure}

加速腔梯度和相位的关系见图\ref{fig:longitudinal}最上方曲线。
当同步相位处于$(-\frac{\pi}{2},0)$之间时,粒子纵向相空间存在势阱,
相稳定区的边界为“鱼形”,如图\ref{fig:longitudinal}所示。
势阱中的粒子围绕同步粒子$(\varphi_s,W_s)$做振荡,能够被稳定的加速。
其稳定区的范围为:
\begin{equation}
    \label{eq:longitudinal_phi}
        \varphi_2 < \varphi < -\varphi_s
\end{equation}
对于小角度震荡,$\varphi_2 \approx 2\varphi_s$,因此稳定区的宽度约为$3|\varphi_s|$。
因此当$|\varphi_s|$增加时,稳定区会增大,但是由于同步相位为负,所以总能量增益会减小。

前面的分析中忽略了阻尼项,但在粒子能量较低时阻尼项不可忽略。当考虑了阻尼项之后,相空间的轨迹也有所变化,但项面积依然保持恒定。
对于低能量粒子的加速,当经过加速后,同步粒子的速度和能量有较大变化,相稳定区将有所增大,
因而加速器的接收范围有所增加,相稳定区域的边界也由原来的“鱼形”变成“螺旋线形”形,如图\ref{fig:longitudinal1}所示。

\begin{figure}[!htb]
    \centering
    \begin{subfigure}[b]{0.48\textwidth}
        \includegraphics[width=\textwidth]{Img/longitudinal1.pdf}
    \end{subfigure}
    \begin{subfigure}[b]{0.45\textwidth}
        \includegraphics[width=\textwidth]{Img/longitudinal2.pdf}
    \end{subfigure}
    \caption{考虑阻尼项后的加速腔电场相稳定区}
    \label{fig:longitudinal1}
\end{figure}


\section{空间电荷效应}
\label{section:spaceChargeIntro}
如小节\ref{section:background}中介绍,空间电荷效应的来源为束团本身。
随着加速器流强的不断提高,束团自身的相互作用与外场对粒子的作用相比已经不可忽略,
甚至在极端流强下,空间电荷对粒子的运动起到了主导作用。

在一个多粒子的束团中,空间电荷效应产生的力可以分为两个方面:
一方面来源于长程的电磁相互作用,即束团中所有粒子在空间中产生了一个近似光滑的电磁场,每一个处于这个电磁场中的粒子都受到作用;
另一方便来源于短程的库伦碰撞作用。
通常来讲,考虑到加速器中的一个束团的粒子数目,相比于短程的库伦碰撞作用,长程的电磁相互作用占据了主导。
所以一般在空间电荷研究中,只考虑长程的相互作用,而忽略短程的碰撞作用。
短程的库伦碰撞引起的散射叫做束内散射(IBS,Intra-Beam Scattering),其不在本文讨论范围内。
在本文中,空间电荷效应指束团内部的长程电磁相互作用。

空间电荷作用与粒子在空间中的分布有关,而粒子的分布又是加速器元件产生的外场和束团自身产生的内场共同作用的结果,
其组成比较复杂,解析计算的话需要联立求解牛顿方程和麦克斯韦方程组(式\ref{eq:Newton}和\ref{eq:Maxwell})。
而对于这组方程,我们目前并不能给出解析解,因此其精确解必须使用数值的方法来得到。

在早期计算机性能不足时,人们使用很多方法对空间电荷效应进行了近似估计,
例如线性近似,在这种方法中,首先假定束团的分布形态,然后使用连续电荷分布所产生的场表达粒子间的相互作用。
例如在三维情况下,如果粒子在实空间中的分布为一个均匀的椭球,则粒子在空间中任意位置的所受到的电场可以表示为\cite{lv2004beamoptic}:
\begin{equation}
    \label{eq:SpaceCharge3D}
    \begin{aligned}
    E_x &= \frac{3ITx}{4 \pi {\varepsilon}_{0} {\gamma}^2 XYZ}{\mu}_x, \\
    E_y &= \frac{3ITy}{4 \pi {\varepsilon}_{0} {\gamma}^2 XYZ}{\mu}_y, \\
    E_z &= \frac{3ITz}{4 \pi {\varepsilon}_{0} {\gamma}^2 XYZ}{\mu}_z
    \end{aligned}
\end{equation}
其中$I$为束流流强,$T$为相邻束团的时间间隔,
$X$,$Y$,$Z$分别是束流在三个方向上的尺寸,$x$,$y$,$z$是粒子的位置,
${\mu}_x$,${\mu}_y$,${\mu}_z$是束团的形状因子:
\begin{equation}
    \label{eq:SpaceCharge3D_mu}
    \begin{aligned}
    {\mu}_x &= \frac{XYZ\gamma}{2}   \int_0^{\infty}
    \frac{d\xi}{(X^2+\xi)            \sqrt{(X^2+\xi)(Y^2+\xi)(Z^2 {\gamma}^2+\xi)}}  \\
    {\mu}_y &= \frac{XYZ\gamma}{2}   \int_0^{\infty}
    \frac{d\xi}{(Y^2+\xi)            \sqrt{(X^2+\xi)(Y^2+\xi)(Z^2 {\gamma}^2+\xi)}}  \\
    {\mu}_z &= \frac{XYZ\gamma}{2}   \int_0^{\infty}
    \frac{d\xi}{(Z^2 {\gamma}^2+\xi) \sqrt{(X^2+\xi)(Y^2+\xi)(Z^2 {\gamma}^2+\xi)}}  \\
    \end{aligned}
\end{equation}

但是由于线性近似方法对束团分布的硬性假设,这种方法得到的解是非自洽的。
而且这种方法只能得到线性空间电荷力的表达式,只能在一定情况下能够和实验结果符合,并不通用。

计算机发展之后,人们使用数值方法对空间电荷效应进行研究。
一种最直接的数值求解空间电荷效应的方法是叠加计算,即对一个粒子逐个计算与其他粒子的相互作用力,然后将其作用力叠加起来,构成该粒子的总空间电荷力。
这一种方法的计算复杂度为$O(N_p^2)$,$N_p$ 是粒子数目。
由于其运算量与粒子数目平方成正比,这种算法在粒子数目较大时的计算开销很大。

另一种求解空间电荷效应的方法为质点网格法(PIC方法),
原理为通过对空间进行网格划分,先将粒子权重到网格上,再在网格上求解泊松方程,
得到空间网格上的电势分布后,再将其作用到单个粒子上。
通过使用网格,PIC的计算复杂度由直接计算的$O(N_p^2)$$N_p^2$ 降低到了$O(\alpha N_p + \beta N_{cells}\log{N_{cells}})$,
其中 $N_p$ 是粒子数,而$N_{cells}$ 是网格点数目,而$\alpha$和$\beta$为算法有关的常数。
由于PIC算法能够有效地降低运算量,绝大多数加速器束流模拟程序使用PIC算法来求解空间电荷力。
关于PIC算法的详细介绍将会在第\ref{section:PIC_algorithm}节中展开。

最近,无网格保辛多粒子追踪算法(Symplectic算法)被引入到加速器研究和模拟中,作为在长距离模拟中空间电荷求解器 \cite{symplectic_ji2017}。
这是因为PIC算法需要将粒子权重到网格上,不可避免的会带来网格热噪声,PIC算法是否可以保障辛条件(symplectic)在目前仍然有较大的争议。
如果不能报障辛条件,那么计算就会被引入一些数值算法带来的非物理的效应。
Symplectic算法并不利用网格,而是利用高阶分解来求解空间电荷效应。
然而,Symplectic算法虽然能够保证辛条件,但其计算量要大得多,计算花费的时间比PIC算法要高两到三个数量级。
幸运的是,无网格算法很适合并行加速运算,有很好的可扩展性。
我们将在后文\ref{section:symplectic_theory}节中对保辛算法的基本原理进行介绍。

\section{国内外强流加速器简介}

在早期,粒子加速器主要服务于高能物理研究,追求更高的能量,加速器发展也主要是朝着更高能量的方向发展。
近年来,强流逐渐成为了另外一个重要的研究方向,
其中一方面是因为某些物理现象需要大量的累积事例,为精确测提供足够高统计度;
另一方面是因为某些极为稀少的事例只有在亮度很大强流加速器才能观察到。
图\ref{fig:proton_ring}展示了正在运行的、建造中的和计划中的高流强高功率质子同步加速器,三条橙色虚线分别代表0.1MW、1MW、10MW的束流功率。
可以看出,受到核物理、高能物理、散裂中子源、加速器驱动的次临界核能系统等应用的驱动,强流质子加速器得到了极大的发展。
\begin{figure}[!htb]
    \centering
    \includegraphics[width=0.7\textwidth]{Img/protonAcc.pdf}
    \caption{高流强高功率质子同步加速器}
    \label{fig:proton_ring}
\end{figure}

表\ref{tab:proton_linac}和\ref{tab:proton_ring}是目前正在运行的、建造中和计划的高功率质子直线加速器和同步加速器的主要参数\cite{tang2011proton}。目前运行的机器主要在1mA量级,功率在0.2-1MW之间。而在建和提出的加速器的功率基本在5MW附近。
下面,我们着重对C-ADS进行介绍。

\begin{table}[!htb]
  \centering
  \begin{tabular}{|>{\small}l|c|c|c|c|c|c|c|}
  %\begin{tabular}{>{\small}l|c|p{0.1\linewidth}|p{0.1\linewidth}|p{0.1\linewidth}|p{0.1\linewidth}|p{0.1\linewidth}|c|c|c}
    \hline
                &Energy &Pulse   &$F_{rep}$ &Duty     &$I_{bun}$ &$I_{ave}$ &$P_{ave}$ \\
                &(GeV)  &length  &(Hz)      &factor   &(mA)      &(mA)      &(MW)      \\
                &       &(ms)    &          &(\%)     &          &          &          \\
    \hline
    LANSCE      &0.8   &0.625    &100/20    &6.2/1.2  &16/9.1    &1.0/0.1   &0.8/0.08  \\
    FNAL        &0.4   &0.05     &15        &0.04     &35        &0.014     &0.007     \\
    SNS         &1.0   &1.0      &60        &6.0      &38        &1.4       &1.4       \\
    J-PARC(1)   &0.18  &0.5      &50/25     &2.5      &50        &0.7       &0.28/0.14 \\
    J-PARC(2)   &0.6   &0.5      &25        &1.25     &50        &0.35      &0.21      \\
    CERN SPL    &2.2   &2.8      &50        &14       &22        &1.8       &4.0       \\
    ESS SP      &1.33  &1.2      &50        &6.0      &114       &3.75      &5.0       \\
    ESS LP      &1.33  &2.0/2.5  &16.67     &4.2      &114/90    &3.75      &5.0       \\
    ESS-S       &2.5   &2.86     &14        &4.0      &50        &2         &5.0       \\
    Project-X   &3/8   &Chopped  &-         &10/2.5   &10        &1/0.25    &3/2       \\
    TRASCO      &$\geqslant 1.0$ &CW  &-      &100      &30        &30        &$\geqslant 30$ \\
    IFMIF       &0.04  &CW       &-         &100      &2*125     &2*125     &10        \\
    C-ADS       &0.15  &CW       &-         &100      &10        &10        &1.5       \\
    \hline
  \end{tabular}
  \caption{国内外强流质子直线加速器}
  \label{tab:proton_linac}
\end{table}

\begin{table}[!htb]
  \centering
  \begin{tabular}{|>{\small}l|c|c|c|c|c|c|}
  %\begin{tabular}{>{\small}l|c|p{0.1\linewidth}|p{0.1\linewidth}|p{0.1\linewidth}|p{0.1\linewidth}|p{0.1\linewidth}|c|c|c}
    \hline
                &Energy &Inject  &$F_{rep}$ &Accu.    &$I_{ave}$ &$P_{ave}$ \\
                &(GeV)  &Energy  &(Hz)      &Part.    &(mA)      &(MW)      \\
                &       &(GeV)   &          &($10^{13}$) &       &          \\
    \hline
    IPNS        &0.45   &0.05    &30        &0.3      &0.167     &0.0075    \\
    ISIS        &0.8    &0.07    &50        &3.75     &0.3       &0.24      \\
    PSR-I       &0.8    &0.8     &20        &3.1      &0.1       &0.08      \\
    PSR-II      &0.8    &0.8     &30        &4.1      &0.2       &0.16      \\
    SNS         &1.0    &1.0     &60        &14.6     &1.4       &1.4       \\
    J-PARC/RCS  &3.0    &0.4     &25        &8.3      &0.333     &1.0       \\
    J-PARC/MR   &50     &3.0     &0.3       &33.2     &0.015     &0.75      \\
    FNAL/Booster&8      &0.4     &15        &7.5      &0.014     &0.12      \\
    ESS         &1.33   &1.33    &50        &2*23.4   &2*1.875   &5.0       \\
    AUSTRON     &1.6    &0.13    &50        &3.9      &0.313     &0.5       \\
    CSNS-I      &1.6    &0.8     &25        &1.56     &0.065     &0.1       \\
    CSNS-II     &1.6    &0.25    &25        &7.8      &0.325     &0.5       \\
    ISNS        &1.0    &0.1     &25        &2.4      &0.1       &0.1       \\
    Project-X/MI&120    &8       &0.4       &16       &0.0167    &2         \\
    \hline
  \end{tabular}
  \caption{国内外强流质子环形加速器}
  \label{tab:proton_ring}
\end{table}


\subsection{C-ADS简介}
加速器驱动次临界洁净核能系统(Accelerator Driven Sub-critical System,ADS),是一种新型的能源装置,其基本原理是利用经过加速器加速的高能质子,撞击重靶核发生反应,一个质子引起的散裂反应可产生几十个中子,再使用散裂产生的中子作为中子源来驱动次临界包层系统,使次临界包层系统维持链式反应以便得到能量和利用多余的中子增殖核材料和嬗变核废物,其中中国ADS(C-ADS)的加速器部分由中国科学院高能物理研究所和近代物理研究所共同研究。

ADS被认为是进行反应堆核废料嬗变的最佳技术方案之一,可以将铀-238转化为更易裂变的钚-239核,或者开发和利用钍资源,其能够充分利用可裂变的核资源,
除此之外,ADS还可以嬗变长寿命核废物为短寿命核废物,降低放射性废物的储量和毒性;
而ADS本身在产能过程中,基本不产生的核废物,可以认为是一种洁净的核能。
另外,ADS是一个次临界系统,需要不断从外界获取中子来维持反应,可以从根本上杜绝事故,其安全性得到保障。

在ADS中,加速器的作用是将质子加速到高能,并通过高能质子束打靶产生中子,从而驱动次临界反应堆持续产生裂变反应。
C-ADS的主要参数如表\ref{tab:C_ADS_parameter}所示\cite{zhihuili2011ADS}。可以看出ADS主要有高流强高功率,CW工作模式,高稳定性等要求。
其中ADS对于束流的短时间中断次数要求非常严格,为了满足这一要求,C-ADS加速器使用了容错设计,即在某些元件失效的情况下,也可以通过一定的补偿,保障机器按照设计的参数正常运行。

\begin{table}[!htb]
  \centering
  \begin{tabular}{>{\small}l|c}
    \hline
    粒子类型              &质子  \\
    \hline
    能量(GeV)           &1.5   \\
    \hline
    流强(mA)            &10    \\
    \hline
    束流功率(MW)        &15    \\
    \hline
    频率(MHz)           &162.5/325/650  \\
    \hline
    占空比(\%)          &100   \\
    \hline
    束损(W/m)           &<1    \\
    \hline
    故障率(每年)        & \\
    \qquad 1s<t<10s        & <25000 \\
    \qquad 10s<t<5m        & <2500  \\
    \qquad t>5m            & <25    \\
    \hline
  \end{tabular}
  \caption{C-ADS加速器主要参数}
  \label{tab:C_ADS_parameter}
\end{table}

因此,C-ADS采用了备份设计,如图\ref{fig:ADS_two_injector}所示\cite{fang2014physics,fang2016instability}。
其中在能量低于10MeV时,采用两条并行的线路,即使用双注入器来保证其可靠性。
两台输入器分别命名为注入器I和注入器II,分别由高能物理研究所和近代物理研究所设计和建造。
在束流能量大于10MeV的加速器,由于其束流能量较高,单腔能量增益对束流运行的状态相比低能段要弱很多,C-ADS主要通过串联备份的方式,即主加速器只有一条线路,但是其工作参数低于其可以达到的正常的运行参数,以留出调整空间。
这样当主加速器中某一个元件失效时,我们可以通过调整其他元件的参数来使整个加速器的输出保持不变。


\begin{figure}[!htb]
    \centering
    \includegraphics[width=\textwidth]{Img/ADS.pdf}
    \caption{C-ADS直线加速器整体布局}
    \label{fig:ADS_two_injector}
\end{figure}


